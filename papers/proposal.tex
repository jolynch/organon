\documentclass[a4paper,10pt]{article}
\usepackage[utf8]{inputenc}

%opening
\title{Symbolic Constraint Language and Solver}
\author{Isaac Evans and Joseph Lynch}

\begin{document}

\maketitle

\section{Introduction}

We propose to built a system licensed under the MIT License for expressing and 
solving complex symbolic constraints between three-dimensional objects. This 
project would significantly expand on the concepts and applications presented in 
the Propagator, Pattern Matching, and Generic Operations aspects of the course.

A potential end use for this system is inclusion into the codebase for MIT’s 
entry to the DARPA Robotics Challenge (DRC). One of our team members is 
currently a student researcher in Prof. Seth Teller’s lab, which is directing 
MIT’s competition entry. Our system will be built with robotic applications in 
mind, but our design is crafted in a way that is careful to avoid any 
architectural dependence on DRC code or conceptual dependency on the idea of 
constraints specific to a robot.

The term “symbolic constraint” implies constraints that are not simply fixed 
offsets. We envision this system being expressive enough to let users author 
constraints such as “these hands grasp the ladder rung at a point not so close 
to the edge that it cannot be reached, but not so close to the center that the 
hands are more than a shoulder-width apart.” Such a constraint should be 
applicable not just to a single ladder instance, but also ladders of varying 
type--with cylindrical/square rungs, or even climbable objects in general such 
as stairs. The goal is that system users can specify behaviors that match their 
“high level conception” of how the constraint should follow in varying 
environments/objects.

In its simplest form, this is an extension from two to three dimensions of some 
high-level GUI frameworks, which offer “fluid” layouts an allow users to 
visualize the widget positioning for an arbitrary window size. However, we plan 
on adding additional complexity which allows for abstract, hierarchical, and 
generative constraints.  Abstract constraints can be specified in a generic 
manner that does not require any information about the specific object below it, 
merely symbolic properties. For instance, we should be able to express an 
constraint idea such as “these bolts are have a position above the surface that 
is a function of both the width of the surface and the proximity of the 
surface to another object.” Hierarchical constraints can be composed with 
boolean operators to create new higher-level constraints. Finally, generative 
constraints are constraint functions that create other partially parameterized 
constraints based on some variables.

Additionally, we will provide mechanisms which allow for a symbolic description 
of several 3D primitive shapes. The variables by which these shapes are 
parameterized, rather than the coordinates of their instantiated vertices, 
will be symbols on which most constraints operate. Taking a cue from 
Aristotle, we will refer to these objects as “forms”.  Forms are the basic data 
structure of our system and are the objects with symbolic meaning that we will 
frame constraints around.

\section{The Language}
In our system, every high level constraint can be described in terms of: 
forms, basic constraints, compound constraints, and generative constraints. 
These concepts go from least general to most general, and obviously it is 
preferred to use simpler concepts if it is possible to explain the constraint 
fully.

\subsection{Form}
Abstract objects that have properties.  These would be implemented as a tagged 
property list.  The tagging allows us to dispatch on form type.

\subsection{Basic Constraint}
A function with no side effects that takes forms as inputs and returns a 
function that can take bindings and results in a floating point state between 
1.0 and 0.0 where 1.0 is fully satisfied and 0.0 is fully not satisfied. If 
needed, context information for why a constraint is in a particular state may 
be provided.  When instantiated, the forms ought be able to fully describe the 
3D vertices of the object (positional information is relative to the origin; 
only the solver has any concept of absolute position). 

\subsection{Compound Constraint} A function with no side effects that takes 
other constraints as inputs and returns a function that combines those lower 
level constraints into a compound output.  These would be implemented via 
simple boolean propagators. When lower levels are instantiated, they will 
propagate up to compound constraints.

\subsection{Generative Constraint}  A function with no side effects that takes 
constraints or forms as inputs and returns a function that is partially 
parameterized by those inputs.  The result would be a function that represents 
the overall constraint tree. This yielded function would not return a state, 
but rather a partially specified constraint tree.

\section{Constraint Propagation System}
Our constraint system will rely on a mechanism similar to how the propagators in 
class work. In particular, when a higher level constraints is created, such 
as a compound or generative constraint, all dependent constraints must be 
informed that they need to propagate changes to that higher level constraint 
upon changes in underlying state.  Basic constraints are responsible for 
informing higher level compound constraints when underlying properties change.  
Convenience functions will be provided that allow changes in the constraint 
propagation to occur during simulation such as the ability to add or remove 
constraints from higher level constraints.  

The simulation will occur within the context of a driver, similar to the drivers 
we have seen in many of the AMB assignments.  This driver would be responsible 
for allowing the user to enter new data or bindings and would drive the 
propagation of that data through the constraint network.  The driver would need 
to keep a lookup table of all basic constraints that rely on particular symbolic 
values, and when the user updates those symbolic values (such as wind 
direction), the driver updates those symbols.  This update will trigger 
propagation to higher level constraints, which will ultimately result 
in a solution as is specified below in the Constraint Solving System section.

There is no notion of time within this constraint propagation system, and there 
are no convergence guarantees, but the code will be written such that the state 
of any given constraint can be inspected at any given moment.

\section{Constraint Solving System}
The input to the solver will be specified as a several constraint functions 
which are parameterized in terms of “forms” from the global scene.

Our solver will support a number of implementations, referred to as “modes,” 
which are somewhat analogous to the layout types of 2D GUI frameworks.

The solver will use the following algorithm:
\begin{enumerate}
 \item Generate inputs either via algorithmic means or static means (see 
stretch goals).
 \item Iteratively apply constraints to the objects in the world. The 
action taken when a constraint fails is implementation/mode specific.  Rely on 
constraint propagation to actually propagate changes through the network.  
Backtracking, if needed, will occur at the constraint level.
 \item Yield for user input.
\end{enumerate}

In “strict” mode, the solver will backtrack as all constraints must be 
satisfied. This is exponential in complexity, so we will provide alternative 
implementations that allow users to specify weights or priorities for 
constraints. Solver modes can be composed so that a subproblem can be solved 
with the strict implementation while another is solved with the weighted 
implementation. 

The final output of this system will be visualized in OpenGL infrastructure 
already developed for the DRC. Care will be taken to ensure that the two systems 
are completely decoupled; the solver might output to a platform-independent 
CAD/modeling file which contains the final vertex lists for the output objects. 
Ideally, however, the output pipeline will be sufficiently fast that constraint 
application and search space pruning can be visualized in real time.

\subsection{Stretch Goals for Solver}
If possible, we would prefer to create our best ``form'' object approximation 
of an input vertex list instead of using static input data.  However, this is 
very difficult and possibly entirely too difficult for the time span.

We would also like it if the solver could answer semantic queries, such as 
``why is box123 at this height?'' with expressions such as ``to satisfy that (1) 
it is above the ground plane (z $>$ 0) and (2) below 50\% of the height of 
cylinder123''.  This is clearly very difficult, and is therefore a stretch goal.

\section{Implementation Plan}
We plan to implement this project in the following steps:
\begin{enumerate}
 \item Determine exactly what kind of constraint relationships we would like to 
support.  If additional types of constraints beyond basic and compound are 
needed, this is when they would be specified.
\item Build a basic test suite that will demonstrate what we eventually hope to 
accomplish at a textual level.  These tests would specify a number of forms in 
a broad range of domains, and show that our constraint propagation system can 
solve useful problems in all of the domains.  At this stage the tests would 
result in purely textual information, no 3D output yet.
\item Build the constraint propagation infrastructure, in particular basic and 
compound constraints and the propagation of new data through the resulting 
constraint tree.
\item Build the constraint driver which allows adhoc changes to the 
constraint tree and data set.
\item Create the solver, which allows multiple modes of operation.  This would 
allow us to start answering more broad semantic queries about the state of the 
system with some degree of certainty without requiring a full exponential 
search. 
\item Create the visualization that results in a 3D “world” represented by 
the solution to the constraints we have imposed.  This would be just a basic 
image or rendering.
\end{enumerate}

Stretch goals are as follows:
\begin{enumerate}
 \item Integration with the actual DRC codebase.
 \item Interactive OpenGL environment.
 \item A NLP parser that would take a human question about why a certain 
constraint is or is not being satisfied and translate it into the symbolic 
language that we have created.
 
\end{enumerate}

Although we think it would be counterproductive to assign specific tasks to 
either group member as both members will be working towards a functional final 
product, the basic division of responsibilities is as follows:

Isaac: Creation of test suite, implementation of generic solver supporting 
multiple modes, implementation of at least a strict and relaxed mode. 

Joey: Creation of test suite, implementation of constraint propagation 
infrastructure, implementation of driver.

\end{document}
